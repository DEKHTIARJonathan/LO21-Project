\documentclass[11pt,a4paper]{article}
\usepackage[UKenglish]{babel}
\usepackage[utf8]{inputenc}
\usepackage{multimedia}
\title{Fort Boyard : Un candidat oublié dans la cellule d’une épreuve retrouvé 7 ans plus tard }
\begin{document}
\maketitle
\section*{Texte: }
La Rochelle – Stupeur et incompréhension pour France Télévision après la découverte de ce week-end. Un homme, ancien candidat de l’émission Fort Boyard, dit avoir été oublié lors d’une épreuve du célèbre jeu dans une des cellules, il y a plus de sept ans. Reportage.

Le visage émacié, amaigri, Aymeric Ledeb revient de loin. L’homme, actuellement hospitalisé au CHU de la Rochelle, n’a pas encore raconté l’intégralité de son histoire aux enquêteurs mais on commence peu à peu à comprendre ce qui s’est passé. « C’était lors d’une épreuve de l’émission enregistrée sur le Fort. Il devait trouver un clé dans une série de jarres remplies de souris, insectes et autres matières visqueuses. Il n’a hélas pas pu terminer à temps et il est resté prisonnier comme le veut la règle », a expliqué un gendarme. Pour une raison jusqu’ici inexpliquée, le reste de ses coéquipiers va alors l’oublier dans sa cellule après la fin de l’émission. « Chacun a pensé qu’il était rentré par ses propres moyens, ou que vexé d’avoir échoué, il ne voulait pas reparler aux autres membres de l’équipe » raconte Ingrid, sa coéquipière de l’époque.

L’enregistrement terminé, toutes les équipes regagnent ensuite le continent, laissant Aymeric à son triste sort. « L’épreuve a été supprimée lors de l’émission suivante et nous avons cessé d’utiliser cette partie du Fort pour les tournages. Personne n’est allé voir dans cette cellule, qui a été oubliée ensuite. Il y a des centaines de cellules de ce type dans tout le Fort » explique pour sa part Colin Jamiel, producteur de l’émission. Les murs très épais du site vont contenir les appels à l’aide de Aymeric. Le jeune homme va survivre miraculeusement en se nourrissant d’araignées, de souris, de rats  et de racines.

Le calvaire d’Aymeric va durer ainsi sept longues années. Jusqu’à ce week-end, quand une équipe chargée de la rénovation de certaines parties du Fort rouvre sa cellule. Ils y découvrent un homme blafard, les cheveux longs, presque aveugle. Pris en charge immédiatement par les secours, l’homme est rapidement hospitalisé. « Pendant qu’on s’occupait de lui, on a remarqué qu’il tenait quelque chose dans sa main, une petite clé » raconte un des pompiers. Ce qui prouverait donc que Aymeric avait donc presque réussi l’épreuve à l’époque.

Pour l’instant l’ancien candidat n’a pas fait part d’une quelconque volonté de poursuivre en justice ses anciens camarades ainsi que la production de l’émission. S’il remportait un tel procès, les propriétaires du Fort pourraient bien débourser une somme astronomique de leur fameux boyards pour réparer le préjudice moral, mettant potentiellement en danger l’avenir de l’émission.


\end{document}
